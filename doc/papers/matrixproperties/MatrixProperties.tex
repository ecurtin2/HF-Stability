\documentclass{revtex4}
\usepackage[margin=1in, paperheight=1in, paperwidth=1in]{geometry}  % doesn't compile w/o width, height
\geometry{
 a4paper,
 total={170mm,257mm},
 left=15mm,
 top=5mm,
}
\usepackage{amsmath}
\usepackage{amssymb}
\usepackage{kbordermatrix}
\usepackage{braket}
\usepackage{xcolor}

\definecolor{myred}{HTML}{CC5803}
\definecolor{myblue}{HTML}{204B57}

\newcommand{\Ap}{\textcolor{myred}{\braket{aj|ib}}}
\newcommand{\App}{\textcolor{myred}{\braket{aj|bi}}}
\newcommand{\Aa}{\textcolor{myred}{\braket{aj||ib}}}
\newcommand{\B}{\textcolor{myblue}{\braket{ab|ij}}}
\newcommand{\Br}{\textcolor{myblue}{\braket{ab|ji}}}
\newcommand{\Ba}{\textcolor{myblue}{\braket{ab||ij}}}

\newcommand{\AtoB}{\mathbf{\alpha\rightarrow\beta}}
\newcommand{\BtoA}{\mathbf{\beta\rightarrow\alpha}}
\newcommand{\AtoA}{\mathbf{\alpha\rightarrow\alpha}}
\newcommand{\BtoB}{\mathbf{\beta\rightarrow\beta}}
\newcommand{\e}{\textcolor{myred}{\left(\epsilon_a-\epsilon_i\right)}}
\newcommand{\diag}{\textcolor{myred}{\delta_{ij}\delta_{ab}}}


\begin{document}
\title{Notes on Matrix Factorization for Hartree-Fock Stability of HEG}
\author{Evan Curtin}
\maketitle

\section{Orbital Hessian Factorization}
According to Seeger and Pople\cite{Seeger1977}, (and many other sources) the molecular orbital
Hessian has the form,
\begin{eqnarray*}
\mathbf{H} =
  \begin{bmatrix}
    \mathbf{A}   & \mathbf{B}   \\
    \mathbf{B^*} & \mathbf{A^*} \\
  \end{bmatrix}
\end{eqnarray*}

Where the matrices denoted by $\mathbf{A}$ and $\mathbf{B}$ are given by,

\begin{eqnarray*}
  A_{st} &=& \e\diag -  \Aa \\
  B_{st} &=& \Ba         \\
\end{eqnarray*}

The color is to help keep track of which portions of the matrices come from
$\mathbf{A}$ and $\mathbf{B}$.
The integration is over spin and spatial coordinates. In the case of
a stationary UHF solution, the matrices $\mathbf{A}$ and $\mathbf{B}$ have the
following forms, after integrating over spin:

\begin{eqnarray*}
  \mathbf{A}&=&\kbordermatrix{
        & \AtoA           & \AtoB           & \BtoA          & \BtoB          \\
  \AtoA & \e\diag + \Aa   & 0               & 0              & \Ap            \\
  \AtoB & 0               & \e\diag - \App  & 0              & 0              \\
  \BtoA & 0               & 0               & \e\diag - \App & 0              \\
  \BtoB & \Ap             & 0               & 0              & \e\diag + \Aa  \\
}
\end{eqnarray*}

\begin{eqnarray*}
  \mathbf{B}&=&\kbordermatrix{
        & \AtoA           & \AtoB           & \BtoA          & \BtoB          \\
  \AtoA & \Ba             & 0               & 0              & \B             \\
  \AtoB & 0               & 0               & -\Br           & 0              \\
  \BtoA & 0               & -\Br            & 0              & 0              \\
  \BtoB & \B              & 0               & 0              & \Ba            \\
}
\end{eqnarray*}
\\
These matrices factorize into ``spin conserved'' $(\mathbf{A', B'})$ and
``spin-unconserved'' $(\mathbf{A'', B''})$ parts, to use the language of Seeger
and Pople. The spin conserved matrices are given by

\begin{eqnarray*}
  \mathbf{A'}&=&\kbordermatrix{
        & \AtoA           & \BtoB          \\
  \AtoA & \e\diag + \Aa   & \Ap            \\
  \BtoB & \Ap             & \e\diag + \Aa  \\
}
\end{eqnarray*}
\begin{eqnarray*}
  \mathbf{B'}&=&\kbordermatrix{
        & \AtoA           & \BtoB          \\
  \AtoA & \Ba             & \B             \\
  \BtoB & \B              & \Ba            \\
}
\end{eqnarray*}
\\
while the spin-unconserved matrices are given by:
\begin{eqnarray*}
  \mathbf{A''}&=&\kbordermatrix{
        & \AtoB           & \BtoA          \\
  \AtoB & \e\diag - \App  & 0              \\
  \BtoA & 0               & \e\diag - \App \\
}
\end{eqnarray*}

\begin{eqnarray*}
  \mathbf{B''}&=&\kbordermatrix{
        & \AtoB           & \BtoA  \\
  \AtoB & 0               & -\Br   \\
  \BtoA & -\Br            & 0      \\
}
\end{eqnarray*}
\\
Thus the spin conserved molecular orbital hessian, $\mathbf{H'}$ is given by:
\begin{eqnarray*}
  \mathbf{H'}&=&\kbordermatrix{
        & \AtoA             & \BtoB            & \AtoA             & \BtoB            \\
  \AtoA & \e\diag + \Aa     & \Ap              & \Ba               & \B               \\
  \BtoB & \Ap               & \e\diag + \Aa    & \B                & \Ba              \\
  \AtoA & \Ba^*             & \B^*             & \e\diag + \Aa^*   & \Ap^*            \\
  \BtoB & \B^*              & \Ba^*            & \Ap^*             & \e\diag + \Aa^*  \\
}
\end{eqnarray*}
\\
and the spin unconserved molecular orbital hessian, $\mathbf{H''}$ is given by:
\begin{eqnarray*}
  \mathbf{H''}&=&\kbordermatrix{
        & \AtoB           & \BtoA              & \AtoB             & \BtoA            \\
  \AtoB & \e\diag - \App  & 0                  & 0                 & -\Br             \\
  \BtoA & 0               & \e\diag - \App     & -\Br              & 0                \\
  \AtoB & 0                 & -\Br^*           & \e\diag - \App^*  & 0                \\
  \BtoA & -\Br^*            & 0                & 0                 & \e\diag - \App^* \\
}
\end{eqnarray*}
\\

This matrix factorizes into to two equivalent matrices,
\begin{eqnarray*}
  \mathbf{H''}&=&\kbordermatrix{
        & \AtoB           &\BtoA            \\
  \AtoB & \e\diag - \App  & -\Br             \\
  \BtoA & -\Br^*          & \e\diag - \App^* \\
}
\end{eqnarray*}

And in this form it is entirely equivalent to the RHF-UHF stability
matrix, $\mathbf{{}^3H'}$ defined in equations 35 and 36 of
Seeger/Pople\cite{Seeger1977}. This is where I suspect I've made a mistake.
\\

\section{Proof of Real-Valued A and B}
In the case of the Homogeneous electron gas,
the two electron integral is given by
(eq. 12 of~\cite{Delyon2008} and p. 16 of~\cite{Guiliani2005}):

\begin{subequations}
\begin{align}
\braket{\vec{k}, \vec{k}'|\vec{k}'',\vec{k} '''}\stackrel{\text{2D, 3D}}{=}&
	\begin{cases}
	\frac{\pi}{\Omega}\frac{2^{D-1}}{|\vec{k}-\vec{k}''|^{D-1}}
	& \vec{k}''' = \vec{k}+\vec{k}'-\vec{k}'' \textbf{ and } |\vec{k}-\vec{k}''| \neq 0 \\
	0
	& \text{else}
	\end{cases}
\\ \nonumber \\
\braket{k, k'|k'',k'''}\stackrel{\text{1D}}{=}&
	\begin{cases}
	e^{|k-k''|^2a^2}\text{Ei}(-|k-k''|^2a^2)\text{;}
	& k''' = k + k'-k'' \textbf{ and } |k-k''| \neq 0 \\
	0\text{;}
	& \text{else}
	\end{cases}
\end{align}
\end{subequations}

The two electron integrals are always real-valued. Therefore
$\mathbf{A} = \mathbf{A}^*$ and $\mathbf{B} = \mathbf{B}^*$

So far I have not used this to simplify anything, but it is true.

\section{References}
\bibliography{../heg_references}
\end{document}
