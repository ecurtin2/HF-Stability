\documentclass{revtex4}
\usepackage[margin=1in, paperheight=1in, paperwidth=1in]{geometry}  % doesn't compile w/o width, height
\geometry{
 a4paper,
 total={170mm,257mm},
 left=15mm,
 top=5mm,
}
\usepackage{amsmath}
\usepackage{amssymb}
\usepackage{kbordermatrix}
\usepackage{braket}
\usepackage{xcolor}
\usepackage{graphicx}

%\definecolor{myred}{HTML}{990000}
%\definecolor{myblue}{HTML}{002299}
\definecolor{myred}{HTML}{000000}
\definecolor{myblue}{HTML}{000000}

\newcommand{\Ap}{\textcolor{myred}{\left(aj|ib\right)}}
\newcommand{\App}{\textcolor{myred}{\left(aj|bi\right)}}
\newcommand{\Aa}{\textcolor{myred}{\left(aj||ib\right)}}
\newcommand{\B}{\textcolor{myblue}{\left(ab|ij\right)}}
\newcommand{\Br}{\textcolor{myblue}{\left(ab|ji\right)}}
\newcommand{\Ba}{\textcolor{myblue}{\left(ab||ij\right)}}

\newcommand{\AtoB}{\mathbf{\alpha\rightarrow\beta}}
\newcommand{\BtoA}{\mathbf{\beta\rightarrow\alpha}}
\newcommand{\AtoA}{\mathbf{\alpha\rightarrow\alpha}}
\newcommand{\BtoB}{\mathbf{\beta\rightarrow\beta}}
\newcommand{\e}{\textcolor{myred}{\left(\epsilon_a-\epsilon_i\right)}}
\newcommand{\diag}{\textcolor{myred}{\delta_{ij}\delta_{ab}}}


\begin{document}
\title{Notes on Matrix Factorization for Hartree-Fock Stability of HEG}
\author{Evan Curtin}
\maketitle

\section{Orbital Hessian Factorization}
According to Seeger and Pople\cite{Seeger1977}, (and many other sources) the molecular orbital
Hessian has the form,
\begin{eqnarray}\label{basic}
\mathbf{H} =
  \begin{bmatrix}
    \mathbf{A}   & \mathbf{B}   \\
    \mathbf{B^*} & \mathbf{A^*} \\
  \end{bmatrix}
\end{eqnarray}
Where the matrices denoted by $\mathbf{A}$ and $\mathbf{B}$ are given by,
\begin{eqnarray}
  A_{st} &=& \e\diag +  \textcolor{myred}{\braket{aj||ib}} \\
  B_{st} &=& \textcolor{myblue}{\braket{ab||ij}}         
\end{eqnarray}
and the two electron integral used here is defined as, 
\begin{eqnarray}
  \braket{pq|rs} &=& \int\int \chi^*_p(1) \chi^*_q(2) \frac{1}{r_{12}} \chi^*_r(1) 
  \chi^*_s(2)
  d\tau d\sigma \\
  \braket{pq||rs} &=& \braket{pq|rs} - \braket{pq|sr} \\
  (pq|rs) &=& \int\int \chi^*_p(1) \chi^*_q(2) \frac{1}{r_{12}} \chi^*_r(1) 
  \chi^*_s(2)
  d\tau \\
  (pq||rs) &=& (pq|rs) - (pq|sr)
\end{eqnarray}
Where $\sigma$ is the spin coordinate and $d\tau$ is the volume element in all spatial coordinates.
Additionally, the convention of $a, b$ being virtual states, $i, j$ being occupied states and 
$p, q, r, s$ being any state is followed.  

The solution is said to be unstable when the lowest eigenvalue is strictly negative. The
corresponding eigenvector points downhill in energy in the case of instability. Defining
the eigenvector as $\mathbf{D}$, the eigenvalue problem can be written as
\begin{eqnarray}
  \mathbf{HD = \lambda D}
\end{eqnarray}
In many cases, the matrix $\mathbf{H}$ factorizes into various components. The following
form of a similarity transformed eigenvalue problem is useful to keep in mind:
\begin{eqnarray}
  \mathbf{S^{-1} H S S^{-1} D} &=& \mathbf{\lambda S^{-1} D} \\
  \mathbf{\tilde H \tilde D} &=& \mathbf{\lambda \tilde D}   
\end{eqnarray} 

Thus if the matrix $\mathbf{H}$ can be transformed via a similarity transformation defined
by $\mathbf{S}$ into $\mathbf{\tilde H}$, then the corresponding eigenvector which satisfies the 
eigenvalue problem
has the form $\mathbf{\tilde D} = \mathbf{S^{-1}D}$.  
Writing the eigenvalue problem in the form of Seeger \& Pople (eq. 18 in \cite{Seeger1977}), 
we get
\begin{eqnarray}
  \begin{bmatrix}
    \mathbf{A} & \mathbf{B} \\
    \mathbf{B^*} & \mathbf{A^*} 
  \end{bmatrix}
  \begin{bmatrix}
    \mathbf{d} \\
    \mathbf{d^*} 
  \end{bmatrix}
  = 2E_2
  \begin{bmatrix}
    \mathbf{d} \\
    \mathbf{d^*} 
  \end{bmatrix}
\end{eqnarray}
We can now apply the similarity transform defined by the Unitary matrix 
\begin{eqnarray}
  \mathbf{U} = \frac{1}{\sqrt{2}}
  \begin{bmatrix}
    \mathbf{I} & -\mathbf{I} \\
    \mathbf{I} & \mathbf{I} 
  \end{bmatrix}
\end{eqnarray}
after which the transformed eigenvalue problem has the form
\begin{eqnarray}
  \frac{1}{2}
  \begin{bmatrix}
    \mathbf{A + B + A^* + B^*} & \mathbf{-A + A^* + B - B^*} \\
    \mathbf{-A + A^* - B + B^*} & \mathbf{A^* + A - B - B^*} 
  \end{bmatrix}
  \begin{bmatrix}
    \mathbf{d + d^*} \\
    \mathbf{d - d^*} 
  \end{bmatrix}
  &=& 2E_2
  \begin{bmatrix}
    \mathbf{d + d^*} \\
    \mathbf{d - d^*} 
  \end{bmatrix} \\
  &=& 2E_2
  \begin{bmatrix}
    \mathbf{Re(d)} \\
    \mathbf{Im(d)} 
  \end{bmatrix}
\end{eqnarray}
If $\mathbf{A}$ and $\mathbf{B}$ are both real, $\mathbf{A = A^*}$ and $\mathbf{B = B^*}$ and the 
above simplifies to 
\begin{eqnarray}
  \begin{bmatrix}
    \mathbf{A + B} & \mathbf{0} \\
    \mathbf{0} & \mathbf{A - B} 
  \end{bmatrix}
  \begin{bmatrix}
    \mathbf{Re(d)} \\
    \mathbf{Im(d)} 
  \end{bmatrix}
  &=& 2E_2   
  \begin{bmatrix}
    \mathbf{Re(d)} \\
    \mathbf{Im(d)} 
  \end{bmatrix}
\end{eqnarray}
Clearly, the matrix factorizes into $\mathbf{A+B}$ and $\mathbf{A-B}$. The eigenvectors factorize
into the real and imaginary displacements of the orbitals, respectively. This is equivalent to the 
condition derived in and around eq. 20 for internal and external instabilities of real
GHF solutions \cite{Seeger1977}, however this approach applies to any matrix with the same
form as equation \ref{basic}.

\section{Forms of equations with UHF wavefunction}
In the case of a stationary UHF solution, the matrices $\mathbf{A}$ and $\mathbf{B}$ have the
following forms, after integrating over spin:
\begin{eqnarray}
  A_{st} &=& \e\diag \textcolor{myred}{+ \delta_{\sigma_i\sigma_a}\delta_{\sigma_j\sigma_b}\Ap
                     - \delta_{\sigma_a\sigma_b}\delta_{\sigma_i\sigma_j}\App} \\
  B_{st} &=& \textcolor{myblue}{\delta_{\sigma_a\sigma_i}\delta_{\sigma_b\sigma_j}\B 
             - \delta_{\sigma_a\sigma_j}\delta_{\sigma_b\sigma_i}\Br} 
\end{eqnarray}
In matrix form, 
\begin{eqnarray}
  \mathbf{A}&=&\kbordermatrix{
        & \AtoA           & \AtoB           & \BtoA          & \BtoB          \\
  \AtoA & \e\diag + \Aa   & 0               & 0              & \Ap            \\
  \AtoB & 0               & \e\diag - \App  & 0              & 0              \\
  \BtoA & 0               & 0               & \e\diag - \App & 0              \\
  \BtoB & \Ap             & 0               & 0              & \e\diag + \Aa  
}
\end{eqnarray}
\begin{eqnarray}
  \mathbf{B}&=&\kbordermatrix{
        & \AtoA           & \AtoB           & \BtoA          & \BtoB          \\
  \AtoA & \Ba             & 0               & 0              & \B             \\
  \AtoB & 0               & 0               & -\Br           & 0              \\
  \BtoA & 0               & -\Br            & 0              & 0              \\
  \BtoB & \B              & 0               & 0              & \Ba            
}
\end{eqnarray}
\\
These matrices factorize into ``spin conserved'' $(\mathbf{A', B'})$ and
``spin-unconserved'' $(\mathbf{A'', B''})$ parts, to use the language of Seeger
and Pople. The spin conserved matrices are given by

\begin{eqnarray}
  \mathbf{A'}&=&\kbordermatrix{
        & \AtoA           & \BtoB          \\
  \AtoA & \e\diag + \Aa   & \Ap            \\
  \BtoB & \Ap             & \e\diag + \Aa  \\
}
\end{eqnarray}
\begin{eqnarray}
  \mathbf{B'}&=&\kbordermatrix{
        & \AtoA           & \BtoB          \\
  \AtoA & \Ba             & \B             \\
  \BtoB & \B              & \Ba            \\
}
\end{eqnarray}
\\
while the spin-unconserved matrices are given by:
\begin{eqnarray}
  \mathbf{A''}&=&\kbordermatrix{
        & \AtoB           & \BtoA          \\
  \AtoB & \e\diag - \App  & 0              \\
  \BtoA & 0               & \e\diag - \App \\
}
\end{eqnarray}

\begin{eqnarray}
  \mathbf{B''}&=&\kbordermatrix{
        & \AtoB           & \BtoA  \\
  \AtoB & 0               & -\Br   \\
  \BtoA & -\Br            & 0      \\
}
\end{eqnarray}
\\
Thus the spin conserved molecular orbital hessian, 
\begin{eqnarray}
\mathbf{H}' =
  \begin{bmatrix}
    \mathbf{A}'   & \mathbf{B}'   \\
    \mathbf{B}'^* & \mathbf{A}'^* \\
  \end{bmatrix}
\end{eqnarray}
is given by
\begin{eqnarray*}
  \mathbf{H'}&=&\kbordermatrix{
        & \AtoA             & \BtoB            & \AtoA             & \BtoB            \\
  \AtoA & \e\diag + \Aa     & \Ap              & \Ba               & \B               \\
  \BtoB & \Ap               & \e\diag + \Aa    & \B                & \Ba              \\
  \AtoA & \Ba^*             & \B^*             & \e\diag + \Aa^*   & \Ap^*            \\
  \BtoB & \B^*              & \Ba^*            & \Ap^*             & \e\diag + \Aa^*  \\
}
\end{eqnarray*}
\\
and the spin unconserved molecular orbital hessian, 
\begin{eqnarray}
\mathbf{H}'' =
  \begin{bmatrix}
    \mathbf{A}''   & \mathbf{B}''   \\
    \mathbf{B}''^* & \mathbf{A}''^* \\
  \end{bmatrix}
\end{eqnarray}
is given by:
\begin{eqnarray*}
  \mathbf{H''}&=&\kbordermatrix{
        & \AtoB           & \BtoA              & \AtoB             & \BtoA            \\
  \AtoB & \e\diag - \App  & 0                  & 0                 & -\Br             \\
  \BtoA & 0               & \e\diag - \App     & -\Br              & 0                \\
  \AtoB & 0                 & -\Br^*           & \e\diag - \App^*  & 0                \\
  \BtoA & -\Br^*            & 0                & 0                 & \e\diag - \App^* \\
}
\end{eqnarray*}
\\
The eigenvalue equation defining $\mathbf{D}''$ is, 
\begin{eqnarray}
  \mathbf{H''D''} = 
  \mathbf{H''}
  \begin{bmatrix}
    \mathbf{d_\AtoB} \\
    \mathbf{d_\BtoA} \\
    \mathbf{d^*_\AtoB} \\        
    \mathbf{d^*_\BtoA} 
  \end{bmatrix}
  = 2E_2
  \begin{bmatrix}
    \mathbf{d_\AtoB} \\
    \mathbf{d_\BtoA} \\
    \mathbf{d^*_\AtoB} \\        
    \mathbf{d^*_\BtoA} 
  \end{bmatrix}
\end{eqnarray}

Transforming the equation with the similarity transform defined by the following Unitary 
matrix, 
\begin{eqnarray}
  \mathbf{U} = \frac{1}{\sqrt{2}}
  \begin{bmatrix}
    \mathbf{I} & 0 & 0 & -\mathbf{I} \\
     0 & \mathbf{I} & -\mathbf{I} & 0 \\ 
     0 & \mathbf{I} & \mathbf{I} & 0 \\
    \mathbf{I} & 0 & 0 & \mathbf{I} \\  
  \end{bmatrix}
\end{eqnarray}
The transformed Hessian is ($e = \e\diag$, $C.C.$ is the complex conjugate of the rest of the 
expression). 
\begin{eqnarray*}
\hspace{-1.45cm}
  &\mathbf{\tilde{H}''}& = \\
  \hspace{-1.45cm} 
  &\frac{1}{2}& \kbordermatrix{
        &           &             &           &          \\
   & e - \App -\Br + C.C.  & 0 & 0  & -e + \App -\Br - C.C.             \\
   & 0               & e - \App -\Br + C.C.     & -e + \App -\Br - C.C. & 0  \\
   & 0 & -e + \App +\Br - C.C. & e - \App +\Br + C.C.  & 0                \\
   & -e + \App +\Br - C.C.    & 0  & 0  & e - \App +\Br +  C.C. \\
}
\end{eqnarray*}
While the corresponding transformed eigenvector has the following form:
\begin{eqnarray*}
  \mathbf{\tilde{D}''} = 
  \begin{bmatrix}
    d_\AtoB + d^*_\BtoA \\
    d_\BtoA + d^*_\AtoB \\
    -d_\BtoA + d^*_\AtoB \\
    -d_\AtoB + d^*_\BtoA \\
  \end{bmatrix}
\end{eqnarray*}

If the two electron integrals and energies are all real, we have for the form 
of $\mathbf{H''}$, 
\begin{eqnarray*}
\hspace{-1cm}
  \mathbf{\tilde{H}''} = 
  \kbordermatrix{
        &           &              &             &      \\
  & \e - \App -\Br  & 0                  & 0                 & 0            \\
  & 0               & \e - \App -\Br     & 0                  & 0  \\
  & 0 & 0 & \e - \App + \Br  & 0                \\
  & 0    & 0  & 0  &\e - \App + \Br \\
}
\end{eqnarray*}
Furthermore, if the solution we started with was an RHF solution, the $\AtoB$ and 
$\BtoA$ are identical. Letting $d_\AtoB = d_\BtoA = d$, we get the following forms 
of the transformed eigenvectors, 
\begin{eqnarray}\label{rhfvecs}
  \mathbf{\tilde{D}''} = 
  \begin{bmatrix}
    d + d^*  \\
    d + d^*  \\
    -d + d^* \\
    -d + d^* \\
  \end{bmatrix}
  =
  \begin{bmatrix}
    Re(d)  \\
    Re(d)  \\
    Im(d)  \\
    Im(d)  \\
  \end{bmatrix}
\end{eqnarray}
At this point the only assumptions we made are that the two electron integrals and the
energies are real-valued and that the solution we started with was an RHF solution. From
this it immediately follows that the eigenvectors have the form of equation \ref{rhfvecs}.

Clearly, the eigensystem defined by $\mathbf{H}''\mathbf{D}'' = 2E_2\mathbf{D}''$ is a
doubly degenerate set of the following equations:
\begin{eqnarray}
  \left[ \mathbf{A+B} \right] Re(d) &=& 2E_2 Re(d) \\
  \left[ \mathbf{A-B} \right] Im(d) &=& 2E_2 Im(d) 
\end{eqnarray}
Where $\mathbf{A}$ and $\mathbf{B}$ are defined such that
\begin{eqnarray}
  A_{ia, jb} &=& \e\diag - \App \\
  B_{ia, jb} &=& -\Br 
\end{eqnarray}
These matrices are exactly equivalent to ${}^3\mathbf{A}'$ and ${}^3\mathbf{B}'$ (eq 35 in Seeger
\& Pople). Thus, when the matrices $\mathbf{A}$ and $\mathbf{B}$ are real, and the solution
is an RHF solution, the eigenvalue problem is doubly degenerate, and factorizes into the equation
characterized by the triplet instability equation. 

\section{Proof of Real-Valued A and B}
In the case of the Homogeneous electron gas,
the two electron integral is given by
(eq. 12 of~\cite{Delyon2008} and p. 16 of~\cite{Guiliani2005}):

\begin{subequations}
\begin{align}
\braket{\vec{k}, \vec{k}'|\vec{k}'',\vec{k} '''}\stackrel{\text{2D, 3D}}{=}&
	\begin{cases}
	\frac{\pi}{\Omega}\frac{2^{D-1}}{|\vec{k}-\vec{k}''|^{D-1}}
	& \vec{k}''' = \vec{k}+\vec{k}'-\vec{k}'' \textbf{ and } |\vec{k}-\vec{k}''| \neq 0 \\
	0
	& \text{else}
	\end{cases}
\\ \nonumber \\
\braket{k, k'|k'',k'''}\stackrel{\text{1D}}{=}&
	\begin{cases}
	e^{|k-k''|^2a^2}\text{Ei}(-|k-k''|^2a^2)\text{;}
	& k''' = k + k'-k'' \textbf{ and } |k-k''| \neq 0 \\
	0\text{;}
	& \text{else}
	\end{cases}
\end{align}
\end{subequations}

The two electron integrals are always real-valued. Therefore
$\mathbf{A} = \mathbf{A}^*$ and $\mathbf{B} = \mathbf{B}^*$

So far I have not used this to simplify anything, but it is true.

\section{References}
\bibliography{../heg_references}
\end{document}
