\documentclass{revtex4}
\usepackage{graphicx}
\usepackage{amsmath}
\usepackage{amssymb}
\usepackage{braket}
\usepackage{mathtools}
\usepackage{textcomp}
\usepackage{algorithm}
\usepackage{booktabs}

\begin{document}
\title{Summary of Hartree-Fock Stability of HEG}
\author{Evan Curtin}
\maketitle


\section{Background}
    The Hartree-Fock procedure has been implemented the quantum many-body problem since its inception and remains the basis for many more advanced techniques even today. In general, the eigenstates of the Hartree-Fock Hamiltonian are solved self-consistently. However, this procedure ensures only that the solution is stationary with respect to the determined orbitals. A method for determining the stability of a Hartree-Fock solution was proposed by Thouless in 1960\cite{Thouless1960}. The condition for stability of a Hartree-Fock solution is equivalent to the conditions for unstable (complex frequency) many-body oscillations within the Random Phase Approximation (RPA). The condition was rederived into the expression familiar to quantum chemists by Čížek and Paldus in 1967\cite{Cizek1967}. Furthermore, the stability equations factorize depending on the symmetry of the Hartree-Fock eigenfunctions. To this end, Seeger and Pople outlined a hierarchical approach to systematically evaluate the stability of HF states in the restricted, unrestricted and generalized Hartree-Fock procedures\cite{Seeger1977}. Recently, the method has been used to aid the in search for the lowest energy Unrestricted Hartree-Fock (UHF) solutions in molecules, as well as the General Hartree-Fock (GHF) solutions in geometrically frustrated hydrogen rings which cannot conform even to the UHF scheme \cite{Pulay2016}\cite{Goings2015}.
    
    The presence of GHF solutions to the Homogeneous Electron Gas which have lower energy than the RHF solutions was proven in the landmark paper by Overhauser\cite{Overhauser1962}. Later still, the ground state energies of the electron gas were found to great accuracy by Ceperley and Alder \cite{Ceperley1980}. In the past few years, phase diagrams have been determined for the HEG in 2 and 3 dimensions\cite{Delyon2008}\cite{Bernu2011}\cite{Baguet2013}. In all cases the phase diagrams are made by computing the energies of the polarized and unpolarized states, and comparing them. The present work focuses on directly calculating the stability of the paramagnetic homogeneous electron gas as a function of electron density. It should be emphasized that instability of the Hartree-Fock solution is a sufficient, but not necessary, condition for the presence of a symmetry-broken Hartree-Fock solution of lower energy to exist. The technique may be used to guide a search for the lowest energy Hartree-Fock solution by following the instability eigenvectors downhill in energy. Such an approach will be attempted in a future paper. 
    
\section{Methods}    
    The Fermi level, $k_f$, is (p. 30 of \cite{Guiliani2005})
    
    \begin{equation}
    k_f
    =\begin{cases} 
    (3\pi^2n)^{\frac{1}{3}} = \left(\frac{9\pi}{4}\right)^{\frac{1}{3}}\frac{1}{r_sa_0} & 3D \\ \\
    (2\pi n)^{\frac{1}{2}} = \frac{\sqrt{2}}{r_sa_0}  & 2D \\
    \\
    \frac{\pi}{2}n = \frac{\pi}{4 r_s a_0}   & 1D 
    \end{cases}
    .
    \end{equation}
    
    The two electron integral is given by (eq. 12 of \cite{Delyon2008} and p. 16 of \cite{Guiliani2005})
    
    \begin{subequations}
    	\begin{align}
    	\braket{\vec{k}, \vec{k}'|\vec{k}'',\vec{k}'''}\stackrel{\text{2D, 3D}}{=}&
    	\begin{cases} 
    	\frac{\pi}{\Omega}\frac{2^{D-1}}{|\vec{k}-\vec{k}''|^{D-1}} 
    	& \vec{k}''' = \vec{k}+\vec{k}'-\vec{k}'' \textbf{ and } |\vec{k}-\vec{k}''| \neq 0 \\
    	0 
    	& \text{else}
    	\end{cases}
    	\\ \nonumber \\
    	\braket{k, k'|k'',k'''}\stackrel{\text{1D}}{=}&
    	\begin{cases} 
    	e^{|k-k''|^2a^2}\text{Ei}(-|k-k''|^2a^2)\text{ ;}
    	& k''' = k+k'-k'' \textbf{ and } |k-k''| \neq 0 \\
    	0\text{ ;} 
    	& \text{else}
    	\end{cases}
    	\end{align}
    \end{subequations}
    where $\Omega$ is the direct lattice volume and D is the dimensionality of the system. Ei(x) 
    denotes the exponential integral function The parameter $a$ in 1D is the radius of a cylinder
    that approximates a 1D system. The limit $a \rightarrow 0$ diverges. The orbital energies are given 
    by 
    
    \begin{equation}\label{eq:hf_orb_energy}
    \epsilon_{\vec{k},\sigma}=
    \frac{\hbar^2\vec{k}^2}{2m} - \sum\limits_{\vec{k}}^{|\vec{k}|< k_f}n_{\vec{k}\sigma}\braket{\vec{k}, \vec{k}' |\vec{k}', \vec{k}}
    \end{equation}
    
    Where $n_{\vec{k}\sigma}$ is the occupation number of the state with momentum $\vec{k}$ and spin $\sigma$. Here we have used the relationship to relate the discrete and continuous quantities, (p. 15 of \cite{Guiliani2005}). 

    \subsection{Hartree-Fock Stability}
    The HF stability condition was presented by Thouless \cite{Thouless1972} ($1^{st}$ Ed. 1960) and is the presence of imaginary electronic oscillation frequencies in the random phase approximation (RPA). Equivalently, instability of a Hartree-Fock solution is borne out as a negative eigenvalue of the electronic Hessian (density-matrix--density-matrix response function). For a paramagnetic HF solution, the equations factorize into the matrices known to chemists as the singlet and triplet instability matrices \cite{Dunning1967}\cite{Seeger1977},
    \begin{equation}\label{eq:1,3H}
    {}^{1,3}\bf{H'}=
    \begin{bmatrix}
    {}^{1,3}\bf{A'} & {}^{1,3}\bf{B'} \\
    \left({}^{1,3}\bf{B'}\right)^* & \left({}^{1,3}\bf{A'}\right)^* \\
    \end{bmatrix},
    \end{equation}
    where the $1,3$ denote singlet and triplet states, respectively. $\bf{A}$, $\bf{B}$ have dimension $N_{occ}\times N_{vir}$ and are defined as follows:
    \begin{subequations}
    	\begin{align}
    	{}^{1}{A'}_{i\rightarrow a, j\rightarrow b} &= (\epsilon_a-\epsilon_i)\delta_{ij}\delta_{ab} + 2\braket{aj|ib}-\braket{aj|bi}\\
    	{}^{3}{A'}_{i\rightarrow a, j\rightarrow b} &= (\epsilon_a-\epsilon_i)\delta_{ij}\delta_{ab} - \braket{aj|bi}\\
    	{}^{1}{B'}_{i\rightarrow a, j\rightarrow b} &= 2\braket{ab|ij}-\braket{ab|ji}\\
    	{}^{3}{B'}_{i\rightarrow a, j\rightarrow b} &= -\braket{ab|ji}
    	\end{align}
    \end{subequations}
    The matrix given in Equation \ref{eq:1,3H} represents the case of complex RHF stability in the space of complex RHF space (${}^{1}\mathbf{H}$ - internal instability) and in the space of complex UHF space (${}^{3}\mathbf{H}$ - triplet instability).

\section{References}
\bibliography{../heg_references}

\end{document}