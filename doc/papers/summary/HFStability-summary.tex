\documentclass{revtex4}
%\usepackage{amsmath}
%\usepackage{amssymb}
%\usepackage{graphicx}

\begin{document}
\title{Summary Hartree-Fock Stability of HEG}
\author{Evan Curtin}
\maketitle


\section{Background}
    The Hartree-Fock procedure has been implemented the quantum many-body problem since its inception and remains the basis for many more advanced techniques even today. In general, the eigenstates of the Hartree-Fock Hamiltonian are solved self-consistently. However, this procedure ensures only that the solution is stationary with respect to the determined orbitals. A method for determining the stability of a Hartree-Fock solution was proposed by Thouless in 1960\cite{Thouless1960}. The condition for stability of a Hartree-Fock solution is equivalent to the conditions for unstable (complex frequency) many-body oscillations within the Random Phase Approximation (RPA). The condition was rederived into the expression familiar to quantum chemists by Čížek and Paldus in 1967\cite{Cizek1967}. Furthermore, the stability equations factorize depending on the symmetry of the Hartree-Fock eigenfunctions. To this end, Seeger and Pople outlined a hierarchical approach to systematically evaluate the stability of HF states in the restricted, unrestricted and generalized Hartree-Fock procedures\cite{Seeger1977}. Recently, the method has been used to aid the in search for the lowest energy Unrestricted Hartree-Fock (UHF) solutions in molecules, as well as the General Hartree-Fock (GHF) solutions in geometrically frustrated hydrogen rings which cannot conform even to the UHF scheme \cite{Pulay2016}\cite{Goings2015}.
    
    The presence of GHF solutions to the Homogeneous Electron Gas which have lower energy than the RHF solutions was proven in the landmark paper by Overhauser\cite{Overhauser1962}. Later still, the ground state energies of the electron gas were found to great accuracy by Ceperley and Alder \cite{Ceperley1980}. In the past few years, phase diagrams have been determined for the HEG in 2 and 3 dimensions\cite{Delyon2008}\cite{Bernu2011}\cite{Baguet2013}. In all cases (to my knowledge) the phase diagrams are made by computing the energies of the polarized and unpolarized states, and comparing them. 

\section{References}
\bibliography{../heg_references}

\end{document}