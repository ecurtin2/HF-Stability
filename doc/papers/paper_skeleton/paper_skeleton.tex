\documentclass{revtex4}
\usepackage{graphicx}
\usepackage{amsmath}
\usepackage{amssymb}
\usepackage{braket}

\newcommand{\ki}{\mathbf{k}_{i}}
\newcommand{\kj}{\mathbf{k}_{j}}
\newcommand{\ka}{\mathbf{k}_{a}}
\newcommand{\kb}{\mathbf{k}_{b}}
\newcommand{\kp}{\mathbf{k}_{p}}
\newcommand{\kq}{\mathbf{k}_{q}}
\newcommand{\kr}{\mathbf{k}_{r}}
\newcommand{\ks}{\mathbf{k}_{s}}

\makeatletter
\makeatother

\begin{document}
\title{Hartree-Fock Stability Analysis of the Paramagnetic Homogeneous Electron Gas in
1, 2 and 3 Dimensions.}
\author{Evan Curtin, So Hirata}
\maketitle

\section{Abstract}

\section{Introduction}

\section{Methods}
We consider the paramagnetic state of the Homogeneous Electron Gas (HEG) in a $D$-dimensional
box of length $L$. All states below the fermi wave vector, $k_f$ are doubly occupied, and all
above are virtual. According to Guiliani and Vignale\cite{Guiliani2005}, $k_f$ is given by equation 
\ref{kf} for 1, 2 and 3 dimensions.

\begin{eqnarray} \label{kf}
k_f
=\begin{cases} 
  (3\pi^2n)^{\frac{1}{3}} = \left(\frac{9\pi}{4}\right)^{\frac{1}{3}}\frac{1}{r_sa_0} & 3D \\ \\
  (2\pi n)^{\frac{1}{2}} = \frac{\sqrt{2}}{r_sa_0}  & 2D \\
  \\
  \frac{\pi}{2}n = \frac{\pi}{4 r_s a_0}   & 1D 
 \end{cases} 
\end{eqnarray}

The Hartree-Fock Hamiltonian  is defined in the usual way for the HEG in that the positive
background charge exactly cancels the electron-electron coulomb interaction. One solution of the 
Hartree-Fock equation is a set of plane waves, 

\begin{eqnarray} \label {planewave}
  \phi_k(\vec{r}) = \frac{ e^ {i \bf{k} \cdot \bf{r}}} {\sqrt{D^N}},  
\end{eqnarray}

defined with periodic boundary conditions\cite{Phillips2012}. Do determine the stability of these
RHF solutions, we use the method introduced by Thouless \cite{Thouless1960}, and further 
chatagerized by Seeger and Pople \cite{Seeger1977}. The solutions of the Hartree-Fock equations are 
unstable when the Molecular Orbital Hessian, 

\begin{eqnarray} \label{H}
  \bf{H} = 
  \begin{bmatrix}
    \bf{A}   & \bf{B} \\
    \bf{B^*} & \bf{A^*}\\
  \end{bmatrix},
\end{eqnarray}

has any strictly negative eigenvalues. An unstable HF solution is one that is a stationary solution
to the HF equations, but has a negative second-order variation in the energy with respect to 
electronic oscillations. Thus, a solution being
unstable is a sufficient condition for another solution of lower energy existing. 

The square matrices $\bf{A}$ and $\bf{B}$ are defined in 
terms of the excitations of the system, and thus have the size of the number of possible 
excitations. Their elements are

\begin{eqnarray} \label{A_B}
	A_{\ki\rightarrow \ka, \kj\rightarrow \kb} &=& 
	(\epsilon_{\ka} - \epsilon_{\ki})\delta_{\ki\kj}\delta_{\ka\kb} + \braket{\ka\kj||\ki\kb}\\
	B_{\ki\rightarrow \ka, \kj\rightarrow \kb} &=& \braket{\ka\kb||\ki\kj},  \nonumber
\end{eqnarray}

where $\epsilon_{\ki}$ is the $i$'th eigenvalue of the one-electron Fock operator. 
The convention for the two electron integrals is, 

\begin{eqnarray} \label{TwoElectron}
  \braket{\kp\kq|\kr\ks}  &=& \int\int \phi_{\kp}^*(\mathbf{x}_1) \phi_{\kq}^*(\mathbf{x}_2) 
                              \frac{1}{r_{12}} 
                              \phi_{\kr}(\mathbf{x}_1) \phi_{\ks}(\mathbf{x}_2) 
                              \mathrm{d}\mathbf{x}_1  \mathrm{d}\mathbf{x}_2, \\
  \braket{\kp\kq||\kr\ks} &=& \braket{\kp\kq|\kr\ks} - \braket{\kp\kq|\ks\kr}                  
\end{eqnarray}

and the integration is over both Cartesian and spin coordinates. The analytic form of the two 
electron integral for the case of the HEG is given by

\begin{subequations} 
  \begin{eqnarray} \label{TwoElectronAnalytic}
    \braket{\kp\kq||\kr\ks}\stackrel{\text{2D, 3D}}{=}
    	\begin{cases} 
      	\frac{\pi}{\Omega} \frac{2^{D-1}}{ |\kp - \kr|^{D-1} } 
      	& \ks + \kr = \kp + \kq  \mathrm{~and~} | \kp - \kr | \neq 0 \\
      	0 
      	& \text{else}
    	\end{cases}
    \\ \nonumber \\
    \braket{\kp\kq||\kr\ks}\stackrel{\text{1D}}{=}
    	\begin{cases} 
      	e^{ |\kp - \kr|^2a^2 }  \text{Ei} (-|\kp - \kr|^2a^2)
      	& \ks + \kr = \kp + \kq \mathrm{~and~} | \kp - \kr | \neq 0 \\
      	0 
      	& \text{else}
    	\end{cases},
  \end{eqnarray}
\end{subequations}

where $\Omega$ is the direct lattice volume. \cite{Delyon2008} \cite{Guiliani2005}. Ei(x) 
denotes the exponential integral function and the parameter $a$ in 1D is the radius of a cylinder
that approximates a 1D system (For more details we refer the reader to reference 
\cite{Guiliani2005}). The eigenenergies of the one particle Fock operator can be determined by
evaluating 

\begin{equation} \label{eq:hf_orb_energy}
  \epsilon_{\kp}=
  	\frac{\hbar^2|\kp|^2}{2m} - \sum\limits_{\ki}^{occ}
    n_{\ki} \braket{\kp, \ki | \ki, \kp}
\end{equation}
Where $n_{\ki}$ is the occupation number of the state $\ki$ and spin \cite{Guiliani2005}. 

A program was implemented to calculate the stability of the HEG with a simple cubic lattice in 1, 2
and 3 dimensions. As we are only interested in the presence of any number of negative eigenvalues, 
it suffices to calculate the lowest eigenvalue and determine its sign. Due to the sparse and 
diagonally dominant nature of $\mathbf{H}$, this is a natural candidate for the Jacobi-Davidson 
Algorithm \cite{Davidson1975}. The algorithm was implented using the SLEPc package, an extension of 
the PETSc suite \cite{Hernandez2005}\cite{petsc-web-page}. 

\section{Discussion}

\section{Conclusion}

\section{Acknowledgements}

\section{References}
\bibliography{paper_skeleton}
\end{document}